\documentclass{article}
\usepackage{graphicx}
\usepackage{lipsum}
\usepackage{changepage}


\begin{document}

\title{Distributed Algorithms Coursework}
\author{Gareth Jones \& Osama Javed}

\maketitle

\begin{abstract}
The following document contains the writeup for our Distributed Algorithms Coursework which focuses on {\em faliure detectors} - specifically in an {\em asynchronous} or {\em partially synchronous} message-passing system with crash faliures.
\end{abstract}

\newpage

\section{Faliure Detectors}

A {\em faliure detector} is a module that provides to each process a collection of {\em suspected} processes.  Detectors in different process may not agree on which processes are suspected.

\subsection{Degrees of Completeness}

{\bf Strong Completeness}\\ 
\begin{adjustwidth}{0.5cm}{}
Every faulty process is eventually permanently suspected by every non-faulty process
\end{adjustwidth}
\vspace{3mm}

\noindent
Weak Completeness\\
\indent 
Every faulty process is eventually permanently suspected by some non-faulty process


\subsection{Degrees of Accuracy}

Strong accuracy - No process is suspected (by anybody) before it crashes
Weak accuracy Some non-faulty process is never suspected

Eventual Strong Accuracy - After some initial period of confusion, no process is suspected before it crashes.  This means no non-faulty process is suspected after some time, since we can take the end of the initial period of chasos as the time at which the last crash occcurs.

Evenrtual weak accuracy - After some initial period of confuasion, some non-faulty process is never suspected.

\subsection{Faliure Detector Classes}

Perfect Detector\\
    Strongly complete and strongly accurate: non-faulty processes are never suspected; faulty processes are eventually suspected by everybody. Easily achieved in synchronous systems.\\ 
    
Strong detector\\
    Strongly complete and weakly accurate. The name is misleading if we've already forgotten about weak completeness, but the corresponding W (weak) class is only weakly complete and weakly accurate, so it's the strong completeness that the S is referring to. \\
    
Eventually Perfect\\
    Strongly complete and eventually strongly accurate. \\
    
Eventually Strong\\
    Strongly complete and eventually weakly accurate.\\

\newpage

%\subsection{Crap}
%\begin{equation}
%    \label{simple_equation}
%    \alpha = \sqrt{ \beta }
%\end{equation}

%\subsection{Subsection Heading Here}
%Write your subsection text here.

%\begin{figure}
%    \centering
%    \includegraphics[width=3.0in]{myfigure}
%    \caption{Simulation Results}
%    \label{simulationfigure}
%\end{figure}

%\section{Conclusion}
%Write your conclusion here.

\end{document}


